\subsection{Программное обеспечение}

Данная работа выполнялась на операционной системе Arch linux c оконным менеджером bspwm. В качестве текстового редактора был использован nvim. Код был написан на языке программирования python с помощью библиотек, которые описаны в таблице \ref{software}.

\begin{table}[H]
    \caption{Используемые библиотеки\label{software}} 

    \begin{tabularx}{1,01\linewidth}{
        | >{\raggedleft\arraybackslash}X
        | >{\raggedright\arraybackslash}m{0.15\linewidth}
        | >{\justifying\arraybackslash}m{0.75\linewidth}|
    }

    \hline
    \rowcolor{clr:1}\multicolumn{1}{|c|}{№} & \multicolumn{1}{c|}{Название} & \multicolumn{1}{c|}{Описание}\\ \hline
    \rowcolor{clr:2}\rownum & pytorch & \noindent платформа глубокого обучения с открытым исходным кодом, доступная с интерфейсом Python и C++ \\ \hline
    \rowcolor{clr:3}\rownum & logging & \noindent модуль определяет функции и классы, которые реализуют гибкую систему ведения журнала событий для приложений и библиотек \\ \hline
    \rowcolor{clr:2}\rownum & ctcdecode & \noindent реализация алгоритма декодирования поиска луча CTC для PyTorch \\ \hline
    \rowcolor{clr:3}\rownum & jiwer & \noindent простой и быстрый пакет python для оценки системы автоматического распознавания речи \\ \hline
    \rowcolor{clr:2}\rownum & tqdm & \noindent библиотека предназначена для внедрения индикаторов выполнения во внешние интерфейсы программ на python \\ \hline
    \rowcolor{clr:3}\rownum & picle & \noindent модуль, который предоставляет возможность сериализовать и десериализовать объекты python \\ \hline
    \end{tabularx}
\end{table}

Для упрощения переносимости кода и необходимого для обучения программного окружения на другие ЭВМ, процесс обучения был проведен с помощью технологии Conda. Данная технология предоставляет готовые пакеты, позволяющие избежать необходимости иметь дело с компиляторами или пытаться понять, как именно настроить конкретный инструмент. 

Среды Conda полезны для обеспечения воспроизводимости биоинформационных проектов. Полная воспроизводимость требует возможности воссоздать систему, которая изначально использовалась для получения результатов. В значительной степени этого можно достичь, используя файл среды Conda для создания среды с определенными версиями пакетов, которые необходимы в проекте.



