\subsection{Процесс обучения}

Для обучения модели данные разбивались на две выборки: тестовую и обучающуюся. К данным выборкам применялся алгоритм фильтрации, описанный выше. После этого программа запускалась в режиме обучения, где после каждой эпохи веса модели сохранялись в отдельный файл. Само обучение заключается в подборе весов методом обратного распространения ошибки, которая высчитывалась с помощью алгоритма CTC loss.

Обучение проводится в течение 200 эпох без использования регуляризации снижения веса, график процесса обучения показан на рисунке \ref{epoch}. Скорость обучения линейно увеличивается в течение первых 1000 шагов до $3e^{-4}$, затем снижается вдвое на эпохах 125, 150 и 175. После обучения нейронная сеть была запущена на тестовой выборке, где показала точность вычисления по WER оценки в 73\%. Результаты предсказаний на тестовой выборки показаны в таблице \ref{example}.

В данной работе также реализованы возможности запуска модели в неско\-льких режимах:

\vspace{0.5em-\topsep}
\begin{itemize}
    \item обучение с нуля;
    \item до обучения уже готовой модели;
    \item тестирование готовой модели обучения.
\end{itemize}

\addimgh{epoch}{1}{График ошибок, полученных во время обучения}{epoch}
