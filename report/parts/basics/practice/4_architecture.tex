\vspace{0.5cm}
\subsection{Архитектура построенной модели}

Разработанная архитектура в данной работе показана на рисунке \ref{architecture}. Она основана на современном подходе, который состоит в использовании совокупности CTC алгоритма и энкодера трансформера. Данную модель можно разделить на две части: обучаемую и алгоритмизированную.

\addimgh{architecture}{0.65}{Архитектура модели распознавания тихой речи}{architecture}

Обучаемая часть состоит из сверточной нейронной сети и трансформера. Этому блоку соответствует листинг кода, показанный на рисунке \ref{code:architecture}. Сверточная нейронная сеть нужна для выделения признаков. На ее вход поступает совокупность временных рядов, которые она преобразует в признаки. После этого полученные признаки поступают на вход к энкодеру трансформера, который получает из них матрицу вероятностей.

Следующая часть, называемая алгоритмизированной, состоит из алгоритма CTC. Данный алгоритм может работать в двух режимах: вычисления ошибки CTC и получении наиболее вероятной последовательности.

\begin{mintedbox}[]{python}{Класс, реализующий архитектуру модели}{code:architecture}
class Model(nn.Module):
    def __init__(self, num_outs):
        super().__init__()
        self.conv_blocks = nn.Sequential(
            ResBlock(8, FLAGS.model_size, 2),
            ResBlock(FLAGS.model_size, FLAGS.model_size, 2),
            ResBlock(FLAGS.model_size, FLAGS.model_size, 2),
        )
        self.w_raw_in = nn.Linear(FLAGS.model_size, FLAGS.model_size)
        encoder_layer = TransformerEncoderLayer(
            d_model=FLAGS.model_size,
            nhead=8,
            relative_positional=True,
            relative_positional_distance=100,
            dim_feedforward=3072,
            dropout=FLAGS.dropout,
        )
        self.transformer = nn.TransformerEncoder(encoder_layer, FLAGS.num_layers)
        self.w_out = nn.Linear(FLAGS.model_size, num_outs)

    def forward(self, x_raw):
        x_raw = x_raw.transpose(1, 2)
        x_raw = self.conv_blocks(x_raw)
        x_raw = x_raw.transpose(1, 2)
        x_raw = self.w_raw_in(x_raw)
        x = x_raw
        x = x.transpose(0, 1)
        x = self.transformer(x)
        x = x.transpose(0, 1)
        return self.w_out(x)
\end{mintedbox}






