\subsection{Сбор данных}

В данной работе входные данные, которые используются для распознания тихой речи, поступают с помощью поверхностной электромиографии или ЭМГ.

Поверхностная ЭМГ использует электроды, помещаемые на верхнюю часть кожи, для измерения электрических потенциалов, вызванных активностью близлежащих мышц. Размещая электроды вокруг лица и шеи, мы можем улавливать сигналы от мышц, важных для речи, которые могут включать челюсть, язык, губы, гортань и мягкое небо. Места размещения электродов описаны в таблице \ref{location}, а визуально их можно увидеть на рисунке \ref{face}. Эти места были выбраны так, чтобы располагаться близко ко многим основным речевым артикуляторам и иметь некоторое сходство.

\begin{table}[H] 
    \caption{Расположение датчиков\label{location}} 
    
    \begin{tabularx}{\textwidth}{|r|>{\centering\arraybackslash}X|}
        \hline
        \rowcolor{clr:1}\multicolumn{1}{|c|}{№} & \multicolumn{1}{|c|}{Местоположение}\\ \hline

        \rowcolor{clr:2}\rownum & левая щека чуть выше рта\hfill \\ \hline
        \rowcolor{clr:3}\rownum & левая часть подбородка \\ \hline
        \rowcolor{clr:2}\rownum & ниже подбородка на 3 см \\ \hline
        \rowcolor{clr:3}\rownum & горло на 3 см левее кадыка \\ \hline
        \rowcolor{clr:2}\rownum & средняя часть челюсти справа \\ \hline
        \rowcolor{clr:3}\rownum & правая щека чуть ниже рта \\ \hline
        \rowcolor{clr:2}\rownum & правая щека на расстоянии 2 см от носа \\ \hline
        \rowcolor{clr:3}\rownum & задняя сторона правой щеки \\\hline
    \end{tabularx}
\end{table}

Используемый набор данных содержит почти 20 часов сигналов ЭМГ лица от одного датчика, записанных со скоростью 1000 выборок в секунду. Сигналы ЭМГ, могут иметь величину до нескольких милливольт, но интенсивность может быть намного ниже в зависимости от силы мышечной активации и расстояния от электродов.

\addimgh{face}{0.6}{Расположение датчиков на лице}{face}
