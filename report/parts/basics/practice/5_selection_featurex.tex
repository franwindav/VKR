\subsection{Алгоритм выделения признаков}

Первый шаг разработанной модели — извлечение признаков из сигналов ЭМГ с минимальной предварительной обработкой. Для этого добавляется набор слоев сверточной нейронной сети в начало модели преобразования, которые будут действовать как экстракторы функций. Эти слои обучаются вместе с остальной частью модели преобразования и дают модели возможность изучать свои собственные признаки ЭМГ сигналов.

Разработанная сверточная архитектура использует стек из трех остаточных блоков свертки, основанный на ResNet, но модифицированный для использования одномерных сверток. Архитектура, используемая для каждого блока свертки, показана на рисунке \ref{conv_architecture}. 

\vspace{1.5em}
\addimgh{conv_architecture}{0.55}{Архитектура сверточного блока}{conv_architecture}

Вдоль основного пути вычислений используются два слоя свертки шириной 3 по последовательности с выпрямленной линейной активацией (ReLU) между ними, а по другому «короткому» пути — одна свертка шириной 1 (линейное преобразование без последовательности) агрегирование. Конечный результат блока представляет собой сумму двух выходных данных пути, за которыми следует активация ReLU.

За каждой сверткой следует операция пакетной нормализации (на рисунке обозначается как БН). Шаги в начале блока установлены равными 2, так что каждый блок сокращается на 2, что дает общее уменьшение длины на 8 по трем слоям. Все свертки имеют размерность канала 768. Этому блоку соответствует листинг кода, показанный на рисунке \ref{code:feature}.

\begin{mintedbox}[]{python}{Класс, реализующий сверточный блок}{code:feature}
class ResBlock(nn.Module):
    def __init__(self, num_ins, num_outs, stride=1):
        super().__init__()
        self.conv1 = nn.Conv1d(num_ins, num_outs, 3, padding=1, stride=stride)
        self.bn1 = nn.BatchNorm1d(num_outs)
        self.conv2 = nn.Conv1d(num_outs, num_outs, 3, padding=1)
        self.bn2 = nn.BatchNorm1d(num_outs)
        if stride != 1 or num_ins != num_outs:
            self.residual_path = nn.Conv1d(num_ins, num_outs, 1, stride=stride)
            self.res_norm = nn.BatchNorm1d(num_outs)
        else:
            self.residual_path = None

    def forward(self, x):
        input_value = x
        x = F.relu(self.bn1(self.conv1(x)))
        x = self.bn2(self.conv2(x))
        if self.residual_path is not None:
            res = self.res_norm(self.residual_path(input_value))
        else:
            res = input_value
        return F.relu(x + res)
\end{mintedbox}

