\subsection{Алгоритм предобработки данных}

Прежде чем приступать к распознанию тихой речи, необходимо сделать достаточно важное мероприятие, которое заключается в предварительной обработке входных данных. В ходе обработки требуется очистить считываемый сигнал $X$ от шумов и посторонних сигналов. Для этого в этой работе используются цифровые фильтры.

Цифровая обработка будет состоять из нескольких фильтров для уменьшения шума и нормализации сигналов. Во-первых, серия режекторных БИХ-фильтров $\text{Ф}_1$  с частотой, кратной 60 Гц, используется для уменьшения шума от сети переменного тока, который значительно проявляется в сигналах перед фильтрацией, часто с величинами, превышающими значения сигналов, которые мы стремимся захватить. Затем используется фильтр высоких частот Баттерворта $\text{Ф}_2$  с частотой среза 2 Гц для удаления постоянного смещения и медленного дрейфа из собранных сигналов, поскольку смещения могут варьироваться в широких пределах и обычно не содержат полезной информации. Для обоих фильтров используется фильтрация вперед-назад, чтобы избежать фазовых сдвигов. Реализованный алгоритм изображен на рисунке \ref{code:preprocessing}.

% \addimgh{preprocessing}{1}{Пример обработки данных}{preprocessing}

Таким образом был определен алгоритм $P$, который проводит предварительную обработку данных ЭМГ сигналов:
\begin{equation}
    P(x) = \text{Ф}_2(\text{Ф}_1(x)) = \hat{x},\hspace{0.5cm} x\in X \text{ и } \hat{x}\in\hat{X}, 
\end{equation}
где $x$ -- считанный сигнал ЭМГ,\\ \phantom{где} $\hat{x}$ -- результат предварительной обработки входных данных. 

\begin{mintedbox}[]{python}{Функции предобработки данных}{code:preprocessing}
    def remove_drift(signal, fs):
        b, a = scipy.signal.butter(3, 2, "highpass", fs=fs)
        return scipy.signal.filtfilt(b, a, signal)

    def notch(signal, freq, sample_frequency):
        b, a = scipy.signal.iirnotch(freq, 30, sample_frequency)
        return scipy.signal.filtfilt(b, a, signal)

    def notch_harmonics(signal, freq, sample_frequency):
        for harmonic in range(1, 8):
            signal = notch(signal, freq * harmonic, sample_frequency)
        return signal 

    def apply_to_all(function, signal_array, *args, **kwargs):
        results = []
        for i in range(signal_array.shape[1]):
            results.append(function(signal_array[:, i], *args, **kwargs))
        return np.stack(results, 1)
\end{mintedbox}

