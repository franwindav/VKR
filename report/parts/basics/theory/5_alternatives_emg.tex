\subsection{Возможные альтернативы замены ЭМГ}

Хотя основное внимание в этой работе будет уделено использованию ЭМГ в качестве входного сигнала для беззвучной речи, это не единственный возможный способ уловить речь, не зависящую от звука. Несколько альтернатив включают визуальные входные данные, электромагнитную артикулографию, ультразвук или сигналы мозга от ЭЭГ, ЭКоГ или фМРТ.

Один из альтернативных методов записи тихой речи — использование видео для визуального чтения по губам говорящего. Некоторыми преимуществами этого метода являются простота записи входных данных и большой объем доступных видеоданных речи, которые можно использовать для обучения. Одним из потенциальных недостатков является то, что видна только внешняя часть лица, а это может означать, что важная информация отсутствует. Есть также компромиссы с простотой использования, поскольку EMG может быть более удобным при ходьбе, а видео удобнее, когда вы сидите перед компьютером.

Еще одним возможным датчиком бесшумного речевого ввода является электромагнитная артикулография, или ЭМА, которая использует магниты, прикрепленные к губам и языку, для отслеживания их движений. Хотя ЭМА обладает очень точной информацией о движении речевых артикуляторов, что делает ее отлично подходит для использования в лаборатории, его необходимость прикреплять предметы к языку делает его слишком инвазионным для многих применений в качестве повседневного устройства связи.

Ультразвуковая визуализация внутренней части рта является еще одним возможным способом улавливания тихой речи. Преимущество ультразвука заключается в том, что он позволяет увидеть язык без установки датчиков во рту. Это может быть менее эффективно при захвате губ, поэтому иногда его комбинируют с визуальными данными для захвата этой информации. Еще одним недостатком является то, что современные ультразвуковые датчики зачастую более дорогие и громоздкие, чем ЭМГ или видео.

Наконец, есть несколько возможных входных данных, основанных на считывании сигналов мозга. Например, датчики ЭЭГ могут считывать электрические сигналы мозга с поверхности кожи точно так же, как датчики ЭМГ считывают мышцы. Однако из-за ослабления сигнала черепом эти датчики могут иметь слишком низкое разрешение, чтобы уловить достаточно информации для декодирования речи. Датчики ECoG, имплантированные внутрь черепа, могут собирать более детальную информацию, хотя для их имплантации требуется хирургическое вмешательство. Методы визуализации, такие как фМРТ, также могут использоваться для захвата речевой информации из мозга, но большой размер и стоимость этих аппаратов делают их непрактичными для многих случаев использования.
