\subsection{Элекромиография}

ЭМГ сигнал, изображенный на рисунке \ref{emg_signal}, представляет собой электрический сигнал, который возникает в мышцах человека как в состоянии покоя, так и при их активации. Этот сигнал измеряется с помощью электромиографии, анализирующей электрическую активность мышц \cite{bib:EMG:3}.

\addimgh{emg_signal}{1}{Электромиографический сигнал}{emg_signal}

Когда мышцы активируются, происходит электрическая активация нейромышечных соединений. Сигнал данной активации показан на рисунке \ref{emg_impulse}. Костный мозг генерирует электрические импульсы, которые приводят к сокращению мышцы. Перед выстрелом мышечные клетки имеют электрический потенциал на внешней мембране из-за дисбаланса заряженных ионов. Когда нервный сигнал заставляет мышцу активироваться, ионные каналы открываются, позволяя ионам проникнуть в клетку, что запускает химический процесс, заставляющий мышцу сокращаться. После этого ионные насосы перемещают ионы обратно через клеточную мембрану. Это движение заряженных ионов в клетку и из нее вызывает распространение электрического импульса, и именно эти импульсы улавливают датчики ЭМГ.

\addimg{emg_impulse}{0.6}{Сигнал активации мышц}{emg_impulse}
