\subsection{Применение тихой речи}

Существует множество различных приложений, в которых может быть полезна тихая речь. Здесь мы обсудим три широкие категории приложений: частное общение, общение с некоторыми формами ограниченной способности говорить и взаимодействие с устройствами \cite{bib:control:1}.

Одна из основных областей применения безмолвной речи — это ввод данных для компьютеров, телефонов и других устройств. Речь может быть очень эффективным методом взаимодействия с устройствами, как за счет использования виртуальных помощников, так и для диктовки текста. Как правило, это намного быстрее, чем другие методы ввода слов, такие как клавиатура, и может быть особенно полезно для мобильных устройств, не имеющих полноразмерной клавиатуры. Однако пользователи могут счесть вокализованную речь неуместной в некоторых условиях из-за соображений конфиденциальности и желания не беспокоить других, и для облегчения этих проблем можно использовать тихую речь.

Еще одним потенциальным вариантом использования безмолвной речи может стать клиническое применение для людей, которые больше не способны произносить нормальную слышимую речь, но все еще используют большую часть своих речевых мышц. Например, это может быть полезно для пациентов, перенесших ларингэктомию, когда гортань была удалена из-за травмы или заболевания. Это также может быть полезно для пациентов с некоторыми типами заболеваний, поражающих нервы или мышцы. Конечно, эффективность в этих случаях будет зависеть от того, насколько целы речевые мышцы и от наличия тренировочных данных.

В зависимости от различных вариантов использования есть два возможных результата, которые мы можем получить от системы бесшумной речи: первый из них это перевод в текстовый формат, второй же – это синтезирование звука. Однако озвучивание можно выполнить косвенно, сначала распознав текст, а затем за тем синтезировав звук с помощью преобразования текста в речь. Поэтому основной задачей будет распознавание текста.
