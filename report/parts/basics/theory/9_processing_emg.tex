\subsection{Обработка входных данных}

\subsubsection{Разметка данных}

Разметка данных играет ключевую роль в процессе первичной обработки информации. В контексте распознавания речи встает сложная проблема разметки речевых данных. Она заключается в том, что входные аудио данные представляют собой непрерывную последовательность звуков, в то время как текст представляет собой дискретную последовательность слов и символов.

Необходимость создания детальной разметки для каждой записи подразумевает определение временных меток для каждого звука или фонемы, что делает этот процесс сложным и затратным. Тем не менее, существуют специальные алгоритмы, такие как "Connectionist Temporal Classification" (CTC), которые предназначены для решения данной проблемы.

Благодаря таким алгоритмам можно более эффективно устанавливать соответствие между записанными звуковыми сигналами и текстовой информацией, что снижает сложность и затраты на процесс разметки данных в задачах распознавания речи.


\subsubsection{Фильтрация входных данных}


Фильтрация биофизических сигналов, как, например, сигналов электромиографии, является важным этапом в обработке данных и требует специального внимания при разработке систем сбора и анализа биофизических сигналов.

Для получения достоверной информации от сигнала ЭМГ необходимо проводить фильтрацию сигнала с целью удаления шумов и помех. Шумы могут быть вызваны различными факторами, такими как электромагнитные помехи, движения пациента, смещение электродов и другие внешние воздействия в процессе сбора данных. Таким образом цель фильтрации - сохранить полезный биологический сигнал, удалив при этом нежелательные помехи, чтобы исследователи могли правильно интерпретировать данные с высокой точностью и достоверностью.

Фильтры можно классифицировать несколькими способами. Для обработки непрерывных во времени сигналов используются аналоговые фильтры, а дискретные во времени и квантовые по амплитуде сигналы обрабатываются цифровыми фильтрами.

Применение аналоговых фильтров происходит в электронике. Они производят антиалиасинговую обработку выводимого изображения, выборку радиостанций в радиоприемниках, разделение звукового сигнала на басы, твитер и др.

Цифровые же фильтры представлены в форме программного и аппаратного обеспечения вычислительных машин \cite{bib:preprocessing:1}. Они используются для спектрального анализа, обработки изображений, видео, речи и звука и др.

Цифровой фильтр можно рассматривать как функцию, которая может быть интерпретирована как во временной, так и в частотной областях. Во временной области она определяет ответ на единичный импульс от устройства или его математической модели, а в частотной области — влияние на амплитуду, фазу на определенной частоте. Переход между этими областями осуществляется с помощью преобразования Фурье, которое устанавливает однозначное соответствие между функциями в обеих областях.

Разделяют два типа цифровых фильтров: фильтры с конечной импульсной характеристикой (КИХ-фильтры) и фильтры с бесконечной импульсной характеристикой (БИХ-фильтры).

Импульсная характеристика — выходной сигнал динамической системы как реакция на входной сигнал в виде дельта функции Дирака, изображенной на рисунке \ref{dirac}, которая отображена на рисунке. В реальных физических системах входной сигнал   представляет   собой   простой   импульс минимальной ширины (равной периоду квантования для дискретных систем) и максимальной амплитуды.

\addimg{dirac}{0.5}{Функция дирака}{dirac}

КИХ и БИХ фильтры отличаются тем, что в КИХ-фильтрах выход зависит только от текущих и прошлых значений входных данных, в то время как в БИХ-фильтрах выход также зависит от предыдущих выходных значений сигнала, помимо входных значений.

Фильтрация осуществляется с использованием различных методов цифровой обработки сигналов, таких как фильтры нижних и верхних частот, фильтры скользящего окна и др.
