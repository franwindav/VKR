\subsection{Помехи сигналов ЭМГ}

При получении биофизических сигналов, содержащих информацию, необходимую для конкретных измерений, дополнительно к основному сигналу также обрабатываются разнообразные шумы и помехи. Различные виды помех классифицируются по своему воздействию на сигнал, по происхождению, по вероятностным характеристикам и по энергетическому спектру.

Некоторые помехи, которые возникают в процессе регистрации сигналов, представляют собой искажения полезных данных, вызванные различными дестабилизирующими факторами, влияющими на измерения. Такие помехи могут быть вызваны, например, промышленным оборудованием или молнией. Помехи классифицируются по различным критериям, таким как их воздействие на сигнал, происхождение, вероятностные характеристики и энергетический спектр.

Источники таких помех подразделяются на внешние и внутренние. Внешние помехи обусловлены электромагнитными волнами, возникающими как в результате действий человека, так и имеющими природное происхождение. Сюда также относятся аппаратурные или инструментальные помехи, а помехами, вызванными деятельностью человека, могут быть, например помехи от переключателей, электродвигателей и других источников. Особое внимание уделяется сетевой помехе как приоритетному внешнему источнику помех.

Основные источники внутреннего шума включают в себя различные физиологические шумы внутри человеческого организма, а также шумы от электродов, связанные с работой других органов.

Типы помех включают в себя импульсные, флуктуационные и периодические помехи. Одной из разновидностей импульсных помех является шумовая помеха, проявляющаяся в виде отдельных импульсов (всплесков) или последовательности случайных импульсов. Источниками импульсных помех являются мгновенные всплески напряжения и тока в транспортных средствах, промышленном оборудовании и природных катаклизмах. Флуктуационные помехи характеризуются хаотическим процессом во времени, представленным непредсказуемыми случайными всплесками различной мощности. Они обычно имеют нормальное распределение с нулевым средним и оказывают существенное воздействие лишь на сигналы низкого уровня. Периодические помехи возникают из-за работы силовых электроустановок и линий электропередач, которые генерируют высокочастотные и низкочастотные поля.

Также искажения от артефактов движения существенно влияют на сигнал. Эти артефакты представляют собой резкие изменения напряжения, которые влияют на сигнал электромиографии (ЭМГ). Причиной появления таких изменений могут быть как движения электродов по поверхности кожи, так и изгибы кабеля, подключенного к электроду. Эти изменения обычно содержат максимальную энергию в диапазоне частот от 0 Гц до 2 Гц.
