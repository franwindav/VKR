\subsection{Системы безмолвного доступа}

Системы безмолвного доступа основаны на использовании сигналов, считываемых с датчиков не подверженных влиянию шума. К таким сенсорам могут относится датчики поверхностной электромиографии.

Исследования, относящиеся к системам безмолвного доступа, проводятся по всему миру. Их популярность обусловлена обширным применением во многих областях, начиная от медицины и заканчивая военной промышленностью.

На данный момент существует разработанная система, основанная на событийном потенциале Р300. Данная система позволяет людям, страдающим от болезней, приводящих к параличу, общаться, считывая по одной букве в слове \cite{bib:EMG:1}.

Подобные нейроинтерфейсы доказали свою эффективность, однако, к тому же они являются медленными, что делает их не пригодными к использованию в повседневной жизни. Также данные устройства требуют понимания их механизма работы, таким образом для их применения нужно достаточно долго учиться и тренироваться.

Как уже упоминалось ранее, системы безмолвного доступа могут применяться в военной индустрии. Данные устройства могут быть применены, когда сообщения необходимо доставить на достаточно большом расстоянии, но при этом другие виды коммуникации могут быть опасны или перехвачены.

Например, управление перспективных исследовательских проектов министерства обороны США в 2008 году выделило 4 миллиона долларов на разработку системы, позволяющей общаться на поле боя без использования вокализированной речи и основанной на использовании сигналов электроэнцефалограммы.

Системы безмолвного доступа могут применятся для широкого спектра задач. В зависимости от поставленной задачи используются необходимые входные данные, которые можно разделить на группы. Дальше рассматриваются наиболее популярные принципы управления системами безмолвного доступа.

\subsubsection{Системы, основанные на распознавании жестов}

Распознавание жестов — это процесс анализа и интерпретации движений человеческого тела с целью соотнесения конкретного жеста с желаемым результатом или командой. Жесты могут включать в себя движения рук, лица или других частей тела.

Для распознавания жестов существует несколько подходов. Один из них использует визуальные признаки, где камеры осуществляют отслеживание и анализ движений. Этот метод позволяет реализовать системы с аутентификацией, вводом информации и управлением, причем для работы такой системы достаточно простой видеокамеры. Однако такой подход имеет свои недостатки, такие как необходимость нахождения человека в кадре и потребность в качественном изображении, что может быть вызовом в некоторых условиях.

Другой подход к распознаванию жестов основан на сборе информации с датчиков, которые регистрируют физическое состояние человека \cite{bib:EMG:2}. Этот метод позволяет выделить более широкий набор характеристик для распознавания жестов. Однако он также влечет за собой определенные сложности, такие как выбор оптимального расположения датчиков на теле человека и создание алгоритмов для обработки шумных сигналов.

\subsubsection{Системы, основанные на распознавании сигналов головного мозга}

Рассмотрим системы, которые реагирует на мысленные намерения человека. Для реализации подобной системы зачастую считывают сигналы ЭЭГ, которые соответствуют воображаемой деятельности, которую мысленно совершает человек. В качестве примера можно привести исследования, которые направлены на анализ сигналов головного мозга для управления курсором мыши, т.е. движений влево или вправо.

Подобные системы довольно сложно реализовать, при этом для этого необходимо специальное дорогое оборудование, что делает его непригодным в повседневной жизни.
