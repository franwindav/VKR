\subsection{Методы регистрации ЭМГ}

В настоящее время существует два вида методов регистрации миоэлектрической активности: инвазивные и неинвазивные. Инвазивный метод, известный как игольчатая электромиография, предполагает введение электрода непосредственно в исследуемую мышцу путем укола. Хотя этот метод обеспечивает наиболее точные результаты, он неудобен для повседневного использования, особенно в области протезирования. Учитывая существенные недостатки инвазивных методов, неинвазивные или поверхностные методы стали более распространенными и предпочтительными.

Поверхностная ЭМГ – это метод записи миоэлектрических сигналов, который использует электроды, прикрепленные к поверхности тела человека. Этот метод более удобен, безболезненен, и позволяет использовать многоразовые электроды для регистрации мышечной активности. Поверхностная ЭМГ нашла широкое применение в медицине, биомеханике, изучении нервной деятельности, реабилитации и лечении двигательных расстройств.
