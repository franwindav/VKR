\subsection{Вводные определения}

В данной работе понятие "тихая речь" означает способ произнесения слов, при котором сводится к минимуму проявление стандартных звуков. Это подразумевает, что артикуляторы перемещаются, но воздушный поток не проходит через них также интенсивно, как в обычной речи. Тихая речь, несмотря на свое название, не является полностью беззвучной из-за возможности издавать звуки при движении губ и языка, но она заметно более тихая по сравнению с обычным произношением.

Для более точного определения тихой речи, стоит провести сравнение с другими типами речи. Существуют разные спектры речи, определяемые своими признаками. Каждая категория включает в себя различные формы речевого проявления, и границы между ними иногда могут быть размытыми, но общими чертами спектра являются четыре различных уровня речевой активности: вокализованная речь, шепот, тихая речь и субвокальная речь.

Вокализация – это «нормальный» речевой режим, используемый для общения в повседневные ситуации. Источником звуков при вокализованной речи является как голос, вызванный активацией голосовых связок, так и другие ограничения потока воздуха в речевом тракте, такие как трение и взрывы.

Следующий режим является шепот. Он отличается от вокализованной речи отсутствием звонкости. При шепотной речи воздух проходит через голосовой тракт, но голосовые связки больше не активируются. Громкость шепотной речи может варьироваться в зависимости от того, с какой силой проталкивается воздух, но обычно она достаточно громкая, чтобы ее услышали другие люди, находящиеся поблизости. Самое тихое проявление шепота называется неслышимым шумом. Несмотря на название, неслышимый шум обычно имеет достаточный поток воздуха, чтобы произвести некоторый звук, но обычно он недостаточно громкий, чтобы его могли понять другие, и его необходимо улавливать с помощью специального стетоскопического микрофона. При неслышимом шуме поток воздуха очень слабый и может быть немного больше, чем поток, возникающий при нормальном дыхании.

Тихая речь — это когда поток воздуха уменьшается до такой степени, что сам поток воздуха не вызывает никакого звука, но речевые артикуляторы все еще двигаются, как при разговоре. Для произнесения тихой речи может потребоваться контроль дыхания, чтобы поддерживать поток воздуха ниже уровня, вызывающего звук.

Последний способ речи — это субвокальная речь, внутреннее появление слов в уме при чтении или мышлении. В субвокальной речи говорящий не делает сознательно никаких попыток пошевелить речевыми артикуляторами. Однако эта внутренняя речь часто сопровождается небольшой активацией речевых мышц, о которой говорящий не осознает. Иногда это включает в себя заметные движения губ, но может также варьироваться от более незаметной активации мышц, которую невозможно увидеть визуально.
