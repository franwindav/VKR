\thispagestyle{empty}
\noindent
\begin{minipage}{0.2\textwidth}
    \includegraphics[width=\linewidth]{mai}
\end{minipage}
\hfill
\begin{minipage}{0.8\textwidth}
    \centering
    \footnotesize
    \textbf{\footnotesize
    МИНИСТЕРСТВО НАУКИ И ВЫСШЕГО ОБРАЗОВАНИЯ\\
    РОССИЙСКОЙ ФЕДЕРАЦИИ}

    \vspace{1em}
    \textbf{\footnotesize
    ФЕДЕРАЛЬНОЕ ГОСУДАРСТВЕННОЕ БЮДЖЕТНОЕ ОБРАЗОВАТЕЛЬНОЕ\\ 
    УЧРЕЖДЕНИЕ ВЫСШЕГО ОБРАЗОВАНИЯ\\ 
    «МОСКОВСКИЙ АВИАЦИОННЫЙ ИНСТИТУТ}\\ 
    (национальный исследовательский университет)»
\end{minipage}

\noindent\rule{\textwidth}{1.5pt}

{
    \small
    \noindent\textbf{Институт \urule\uline{«Компьютерные науки и прикладная математика»}\urule\ \ Кафедра \urule\uline{805}\urule}\\
    \noindent\textbf{Группа \urule\uline{М8О-403Б-20}\urule\ \ Направление подготовки \urule\uline{01.03.04 «Прикладная математика»}\urule}\\ 
    \noindent\textbf{Профиль \uline{\ «Математическое и программное обеспечение систем обработки информации и управления»}\urule}\\ 
    \noindent\textbf{Квалификация \uline{\ \ \ \ бакалавр}\urule}
}

\vspace{0.5em}
\noindent\hfill
\begin{minipage}{0.65\textwidth}
    \small

    \begin{center}
        \textbf{УТВЕРЖДАЮ}
    \end{center}
    
    \vspace{-0.8em}
    \noindent Заведующий кафедрой $\underset{\text{№ каф.}}{\text{\uline{\ \ 805\ \ }}}$
        \hspace{0.2cm}
        $\underset{\text{подпись}}{\text{\uline{\hspace{2cm}}}}$
        \hspace{0.2cm}
        $\underset{\text{ФИО}}{\text{\uline{А.В. Пантелеев}}}$
    
    \vspace{0.5em}
    \noindent\hfill\uline{\ \ «09»\ \ }\ \ \ \uline{\ \ февраля\ \ }\ \ \ \uline{\ \ 2024 г.\ \ }
\end{minipage}

\vspace{0.3em}
\begin{center}
    \changefontsize[27]{18}
    \noindent\textbf{ЗАДАНИЕ}
    
    \vspace{-0.8em}
    \noindent\textbf{на выпускную квалификационную работу бакалавра}
\end{center}

\vspace{-0.5em}
{
    \small
    \noindent\textbf{Обучающийся} 
    \urule$\underset{\text{ФИО полностью}}{\text{\uline{Большаков Максим Владимирович}}}$\urule

    \noindent\textbf{Руководитель}
    \urule$\underset{\text{ФИО полностью}}{\text{\uline{Алексейчук Андрей Сергеевич}}}$\urule

    \vspace{-0.7em}
    \noindent\vphantom{A}\urule$\underset{\text{ученая степень, учебное звание, должность и место работы}}{\text{\uline{\hphantom{22}к.ф.-м.н., доцент, доцент каф. 805 МАИ\hphantom{22}}}}$\urule

    \noindent\textbf{1. Наименование темы} \uline{\ \ «Распознавание сигналов электромиографии для устройств без\-мол\-в\-но\-го доступа»}\urule


    \noindent\textbf{2. Срок сдачи обучающимся законченной работы} \urule\uline{23.05.2024}\urule
    
    \noindent\textbf{3. Задание и исходные данные к работе}

    { 
        \noindent\uline{Исходными данными являются сигналы электромиографии речевых артикуляторов, пре\-дста\-в\-лен\-ные в виде временных рядов, а также соответствующая им текстовая информация. На основе исходных данных необходимо разработать и построить модель машинного обу\-че\-ния, способную переводить тихую речь в текстовый формат. Обученная модель должна иметь точность, позволяю\-щую дальнейшую интеграцию в устройства безмолвного доступа.}\urule\\ 
    }
   
    \vspace{-0.5em}
    \noindent\textbf{Перечень иллюстративно-графических материалов:}
    
    \noindent
    \renewcommand{\tabularxcolumn}[1]{m{#1}}
    \begin{tabularx}{\linewidth}{
        | >{\centering\arraybackslash\footnotesize}m{0.03\linewidth}
        | >{\justifying\arraybackslash\footnotesize}X
        | >{\centering\arraybackslash\footnotesize}m{0.2\linewidth}|
    }
        \hline
        \textbf{№ п/п} & \multicolumn{1}{|c|}{\footnotesize \textbf{Наименование}} & \textbf{Количество листов}\\
        \hline
        1 & \noindent Раздаточный материал & 15 \\
        \hline
    \end{tabularx}



    \newpage
    \thispagestyle{empty}
    \noindent\textbf{4. Перечень подлежащих разработке разделов и этапы выполнения работы}

    \noindent
    \renewcommand{\tabularxcolumn}[1]{m{#1}}
    \begin{tabularx}{\linewidth}{
        | >{\centering\arraybackslash\footnotesize}m{0.03\linewidth}
        | >{\justifying\arraybackslash\footnotesize}X
        | >{\centering\arraybackslash\footnotesize}m{0.2\linewidth}
        | >{\centering\arraybackslash\footnotesize}m{0.18\linewidth}
        | >{\centering\arraybackslash\footnotesize}m{0.12\linewidth} |
    }
        \hline
        № п/п & \multicolumn{1}{|c|}{\footnotesize Наименование раздела или этапа} & Трудоемкость в \% от полной трудоемкости ВКРБ & Срок выполнения & Примечание\\ 
        \hline
        1 & \noindent Изучение и анализ материала по сопутствующей области. & 15\% & 09.02.24 -- 28.02.24 &  \\
        \hline
        2 & \noindent Получение и подготовка набора данных для обучения модели. Оценка качества сигналов, идентификация шума или артефактов. & 25\% & 01.03.24 -- 19.03.24 &  \\
        \hline
        3 & \noindent Создание и обучение модели машинного обучения на подготовленных данных. & 25\% & 20.03.24 -- 04.04.24 &  \\
        \hline
        4 & \noindent Оценка модели на тестовом наборе данных, анализ результатов. Выводы о работоспособности модели, ее точности и эффективности. & 15\% & 05.04.24 -- 24.04.24 &  \\
        \hline
        5 & \noindent Подведение итогов и обобщение результатов исследования. & 10\% & 25.04.24 -- 14.05.24 &  \\
        \hline
        6 & \noindent Оформление выпускной квалификационной работы. & 10\% & 15.05.24 -- 24.05.24 &  \\
        \hline

    \end{tabularx}

    \noindent\textbf{5. Исходные материалы и пособия}
    {
        \sloppy
        \footnotesize
        \vspace{-\topsep}
        \begin{itemize}[parsep=0.4em]
            \item[1.] \uline{
                    Vaswani A., Shazeer N., Parmar N., Uszkoreit J., Jones L. Attention Is All You Need [Electronic resource]. // arXiv preprint arXiv:1706.03762. --- 2023. --- URL: \url{https://arxiv.org/pdf/1706.03762} (date of treatment: 05.02.2024).}\urule
            \item[2.] \uline{Chorowski J. Attention-based models for speech recognition. // Advances in neural information processing systems. --- 2015. --- P. 577-585.}\urule 
            \item[3.] \uline{Pascanu R., Mikolov T., Bengio Y. Understanding the exploding gradient problem. // CoRR, abs/1211.5063. --- 2012. --- Vol. 2, №417. --- P. 1.}\urule
            \item[4.] \uline{Шелухин О.И., Лукьянцев Н.Ф. Цифровая обработка и передача речи: Учебное пособие. --- М.: Радио и связь, 2000. --- С. 454.}\urule
            \item[5.] \uline{А. С. Колоколов. Обработка сигнала в частотной области при распознавании речи. // Проблемы управления. --- 2006. --- С. 13–18.}\urule
            \item[6.] \uline{Amodei D., Anubhai R., Battenberg E., Case C. Deep Speech 2: End-to-End Speech Recognition in English and Mandarin [Electronic resource]. // arXiv preprint arXiv:1512.02595. --- 2015. --- URL: \url{https://arxiv.org/pdf/1512.02595} (date of treatment: 10.03.2024). }\urule
        \end{itemize}
    }
    \noindent\textbf{6. Дата выдачи задания} \urule\uline{09.02.2024}\urule
    
    \vspace{2.5em}
    \noindent\hfill
    \begin{minipage}{0.45\textwidth}
        \small
        \noindent Руководитель
            \hspace{0.2cm}
            \urule$\underset{\text{подпись}}{\text{\uline{\hspace{2cm}}}}$\urule
        
        \vspace{1em}
        \noindent Задание принял к исполнению 
            \hspace{0.2cm}
            \urule$\underset{\text{подпись}}{\text{\uline{\hspace{2cm}}}}$\urule
    \end{minipage}
    \hspace{0.2cm}
    \begin{minipage}{0.2\textwidth}
        \small
        \noindent\urule$\underset{\text{ФИО}}{\text{\uline{Алексейчук А.С.}}}$\urule

        \vspace{1em}
        \noindent\urule$\underset{\text{ФИО}}{\text{\uline{Большаков М.В.}}}$\urule
    \end{minipage}

}
