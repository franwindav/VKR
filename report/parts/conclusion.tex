\anonsection{Заключение}

Основной результат выпускной квалификационной работы заключается в разработке и построении с использованием современных технологий модели машинного обучения, способной переводить электромиографические сигналы речевых артикуляторов в текстовый формат. 

Архитектура созданной модели основана на использовании сверточных нейронных сетей для расширения признакового пространства, а также на применении алгоритма CTC в сочетании с энкодером трансформера.
Помимо этого в разработанной модели была использована 5-граммная языковая модель, что позволило повысить точность предсказания слов в предложении, а также снизить количество ошибок в словах.

\noindent\textbf{В ходе выполнения данной работы были выполнены следующие задачи:}

\vspace{0.2em-\topsep}
\begin{itemize}
    \item рассмотрены подходы к обработке и анализу сигналов ЭМГ;
    \item проведен обзор методов распознавания речи;
    \item исследованы методы выделения признаков из сигналов ЭМГ;
    \item разработан алгоритм предварительной обработки сигналов ЭМГ;
    \item построена и обучена модель машинного обучения для распознавания тихой речи;
    \item проведен анализ полученных результатов.
\end{itemize}


