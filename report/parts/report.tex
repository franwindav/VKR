\setcounter{page}{2}
{\centerline{\uppercase{Реферат}}}

Выпускная квалификационная работа бакалавра содержит 70 страниц, 24 рисунка, 3 таблицы, 17 использованных источников.

\uppercase{
    машинное обучение, распознавание речи, электромиография, коннекционистская временная классификация, механизм внимания, трансформер, языковая модель, нейронные сети
}

В выпускной квалификационной работе бакалавра показано решение задачи распознавания тихой речи с использованием в качестве 
входных данных сигналов электромиографии речевых артикуляторов. В рамках этой задачи используются цифровые фильтры для предобработки входных данных.
Архитектура модели машинного обучения базируется на алгоритме коннекционистской временной классификации
и нейронной сети трансформер.

На основе предложенных методов была разработана и обучена модель машинного обучения, способная переводить тихую речь в текстовый формат.
Полученные результаты демонстрируют перспективность разработанной системы для распознавания сигналов электромиографии.

