\anonsection{Введение}


В современном мире, где технологии становятся все более важными в нашей жизни, появляются новые требования к удобству и эффективности управления устройствами. Одним из перспективных и инновационных подходов к решению этой задачи является создание систем безмолвного доступа, которые позволяют пользователю манипулировать устройствами только с помощью сигналов, генерируемых мышцами.

Электромиография (ЭМГ) – это метод изучения электрической активности мышц, который позволяет записать электрические сигналы, генерируемые мышцами при их сокращении. Применение технологий электромиографии для управления устройствами открывает широкие перспективы, особенно для людей с физическими ограничениями или в условиях, когда голосовое или визуальное управление невозможно или неудобно.

В большинстве случаев под распознаванием речи понимается преобразование аудио-последовательности записи голоса человека в текстовые данные. Однако, в некоторых случаях использование не только звуковой, но и другой информации позволяет улучшить модель или даже полностью заменить аудио-модель. Таким образом распознавание сигналов электромиографии для устройств безмолвного доступа можно рассматривать как задачу распознавания речи, но с уникальными способностями.

По аналогии с тем, как распознается устная речь человека, распознание сигналов электромиографии речевых артикуляторов требует анализа и интерпретации мимических мышечных движений, связанных с произношение звуков и слов. 

Однако распознание сигналов речевых артикуляторов имеет свои уникальные аспекты. В отличие от анализа звуков речи, который основан на акустических сигналах, распознавание сигналов ЭМГ требует работы с электрическими сигналам, сгенерированными лицевыми и шейными мышцами. Это означает необходимость использования специализированных методов обработки и алгоритмов для анализа и интерпретации этих сигналов. 

Целью данной дипломной работы является исследование и разработка методов распознавания сигналов электромиографии для возможности создания систем безмолвного доступа. Для достижения этой цели необходимо: 

\vspace{0.5em-\topsep}
\begin{itemize}
    \item провести обзор существующих подходов к обрабоке и анализу данных сигналов электромиографии;
    \item разработать алгоритма предобработки данных сигналов электромиографии на основе анализа исходного набора данных;
    \item исследовать методы обогащения признакового пространства для распрознавания тихой речи;
    \item провести обзор методов распознаванию речи;
    \item построить и обучить модель машинного обучения, способную переводить тихую речь в текстовый формат;
    \item провести анализ обученной модели машинного обучения.
\end{itemize}

Полученные результаты могут быть использованы не только для создания системы безмолвного доступа, способной взаимодействовать с различными устройствами, но и в областях медицины и реабилитации.

