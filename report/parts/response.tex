\thispagestyle{empty}
\noindent
\begin{minipage}{0.2\textwidth}
    \includegraphics[width=\linewidth]{mai}
\end{minipage}
\hfill
\begin{minipage}{0.8\textwidth}
    \centering
    \footnotesize
    \textbf{\footnotesize
    МИНИСТЕРСТВО НАУКИ И ВЫСШЕГО ОБРАЗОВАНИЯ\\
    РОССИЙСКОЙ ФЕДЕРАЦИИ}

    \vspace{1em}
    \textbf{\footnotesize
    ФЕДЕРАЛЬНОЕ ГОСУДАРСТВЕННОЕ БЮДЖЕТНОЕ ОБРАЗОВАТЕЛЬНОЕ\\ 
    УЧРЕЖДЕНИЕ ВЫСШЕГО ОБРАЗОВАНИЯ\\ 
    «МОСКОВСКИЙ АВИАЦИОННЫЙ ИНСТИТУТ}\\ 
    (национальный исследовательский университет)»
\end{minipage}

\noindent\rule{\textwidth}{1.5pt}

\vspace{-2em}
\noindent
\begin{center}
    \changefontsize[24]{16}
    \noindent\textbf{ОТЗЫВ}

    \changefontsize[21]{14}
    \vspace{-0.8em}
    \noindent\textbf{РУКОВОДИТЕЛЯ}
\end{center}

\vspace{-0.5em}
{
    \small
    \noindent\textbf{Обучающийся} \urule\uline{Большаков Максим Владимирович}\urule\\
    \noindent\textbf{Институт} \urule\uline{«Компьютерные науки и прикладная математика»}\urule\ \ \textbf{Кафедра} \urule\uline{805}\urule\\
    \noindent\textbf{Группа} \urule\uline{М8О-403Б-20}\urule\ \ \textbf{Направление подготовки} \urule\uline{01.03.04 «Прикладная математика»}\urule\\ 
    \noindent\textbf{Профиль} \uline{\ \ \ \ «Математическое и программное обеспечение систем обработки информации и уп\-рав\-ле\-ния»}\urule\\ 
    \noindent\textbf{Квалификация} \uline{\ \ \ \ бакалавр}\urule\\ 
    \noindent\textbf{Наименование темы} \uline{\ \ \ \ «Распознавание сигналов электромиографии для устройств без\-мол\-в\-но\-го доступа»}\urule\\  
    \noindent \textbf{Руководитель} 
        \urule$\underset{\text{ФИО полностью}}{\text{\uline{Алексейчук Андрей Сергеевич}}}$\urule 
    
    \vspace{-0.6em}
    \noindent\vphantom{A}\urule$\underset{\text{ученая степень, учебное звание, должность и место работы}}{\text{\uline{\hphantom{22}к.ф.-м.н., доцент, доцент каф. 805 МАИ\hphantom{22}}}}$\urule


    \vspace{0.2cm}

    \noindent
    \vphantom{i}
    \uline{
        \ \ \ \ В результате выполнения выпускной квалификационной работы Большаковым М.В. была разработана и построена модель машинного обучения способная переводить сигналы эле\-ктро\-мио\-гра\-фии речевых артикуляторов в текстовый формат. Обученная модель, выдававшая 73\% точности, была основана на алгоритме коннекционистской временной классификации и нейронной сети трансформер.
        Выполненная работа может быть использована не только для создания системы безмолвного доступа, способной взаимодействовать с различными устройствами, но и в областях медицины и реабилитации.
    }\urule\\ 
    \noindent
    \vphantom{i}
    \uline{
        \ \ \ \ Во время выполнения ВКРБ Большаков М.В. продемонстрировал знания, умения и навыки, соответствующие предметной области, а также самостоятельность в выборе ин\-стру\-мен\-тов для решения поставленной задачи. Материалы, изложенные в ВКРБ, полностью соответству\-ют ин\-ди\-ви\-ду\-аль\-но\-му заданию. Задачи, поставленные в ходе работы, были вы\-пол\-не\-ны, а по\-лу\-чен\-ные результаты имеют практическую ценность и могут быть использова\-ны в реальных проектах.
    }\urule
    \noindent
    \vphantom{i}
    \uline{
        \ \ \ \ Работа проверена на объем заимствования. \% заимствования -- 13.5\%.
    }\urule

    \noindent Заключение: \uline{Большаков Максим Владимирович за выполнение выпускной ква\-ли\-фи\-ка\-цион\-ной работы заслуживает оценку 5 (отлично) с присуждением степени бакалавра по на\-пра\-вле\-нию «При\-кла\-дная математика».}\urule

    \noindent\uline{\ \ «23»\ \ }\ \ \ \uline{\ \ мая\ \ }\ \ \ \uline{\ \ 2024 г.\ \ } 
    \hfill
    Руководитель $\underset{\text{подпись}}{\uline{\hspace{2.5cm}}}$ Алексейчук А.С.

    \vfill

    \begin{center}
        Москва 2024
    \end{center}
}
