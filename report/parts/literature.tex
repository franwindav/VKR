\anonsection{Список использованных источников}

\begingroup
    \renewcommand{\section}[2]{}%
    \begin{thebibliography}{}

        \bibitem{bib:control:1}
            Kobrinski A., Bolkovitin S., Voskoboinikova L. Problems of bioelectric control. // Automatic and Remote Control : proc. 1st IFAC Int. Congress. --- 1960. --- Vol. 2. --- P. 619.

        \bibitem{bib:EMG:1}
            Ганин И.П., Каплан А.Я. Интерфейс мозг компьютер на основе волны p300: предъявление комплексных стимулов “подсветка + движение”. // Журнал высшей нервной деятельности им. И. П. Павлова. --- 2014. --- Т. 64, № 1. --- С. 32--40.

        \bibitem{bib:EMG:2}
            Унанян Н.Н., Белов А.А. Распознавание мышечной активности с помощью электромиографических датчиков в задачах управления бионическим механизмом. // Материалы 15-й Международной конференции «Устойчивость и колебания нелинейных систем управления» (конференция Пятницкого). --- М.: ИПУ РАН, 2020. --- С. 427--430.

        \bibitem{bib:EMG:3}
            Будко Р.Ю., Чернов Н.Н., Будко А.Ю. Распознавание мышечных усилий по сигналу лицевой электромиограммы в режиме реального времени. // Приборостроение, метролония и информационно-измерительные приборы и системы. Научный вестник НГТУ. --- 2018. --- Т. 71, № 2. --- С. 59–74.

        \bibitem{bib:preprocessing:1}
            Шелухин О.И., Лукьянцев Н.Ф. Цифровая обработка и передача речи: Учебное пособие. --- М.: Радио и связь, 2000. --- 454 с.

        \bibitem{bib:feature:1}
            А. С. Колоколов. Обработка сигнала в частотной области при распознавании речи. // Проблемы управления. --- 2006. --- С. 13–18.
            
        \bibitem{bib:feature:2}
            Жиляков Е.Г., Белов С.П., Прохоренко Е.И. Вариационные методы частотного анализа звуковых сигналов. // Труды учебных заведений связи. --- СПб, 2006. --- № 174. --- С.163-170. 
    
        \bibitem{bib:gradient:1}
            Pascanu R., Mikolov T., Bengio Y. Understanding the exploding gradient problem. [Electronic resource]. // arXiv preprint arXiv:1211.5063. --- 2012. --- URL: \url{https://arxiv.org/pdf/1211.5063} (дата обращения: 07.02.2024). 
 
        \bibitem{bib:speech:1}
            Винцюк Т.К. Анализ, распознавание и интерпретация речевых сигналов. // Институт кибернетики им. В. М. Глушкова. --- Киев: Наукова думка, 1987. --- C. 264.

        \bibitem{bib:end-to-end:1}
            Hannun A., Case C., Casper J. Deep speech: Scaling up end-to-end speech recognition [Electronic resource]. // arXiv preprint arXiv:1412.5567. --- 2014. --- URL: \url{https://arxiv.org/pdf/1412.5567} (дата обращения: 07.02.2024). 

        \bibitem{bib:end-to-end:2}
            Amodei D., Anubhai R., Battenberg E., Case C. Deep Speech 2: End-to-End Speech Recognition in English and Mandarin [Electronic resource]. // arXiv preprint arXiv:1512.02595. --- 2015. --- URL: \url{https://arxiv.org/pdf/1512.02595} (дата обращения: 10.03.2024). 

        \bibitem{bib:CTC:1}
            Graves A. Connectionist temporal classification: labelling unsegmented sequence data with recurrent neural networks // Proceedings of the 23rd international conference on Machine learning. --- 2006. --- P. 369--376.

        \bibitem{bib:CTC:2}
            Hori T., Watanabe S., Zhang Y., Chan W. Advances in joint CTC-attention based end-to-end speech recognition with a deep CNN encoder and RNN-LM [Electronic resource]. // arXiv preprint arXiv:1706.02737. --- 2017. --- URL: \url{https://arxiv.org/pdf/1706.02737} (дата обращения: 1.09.2023). 

        \bibitem{bib:LSTM:1}
            Sak H., Senior A., Beaufays F. Long short-term memory recurrent neural network architectures for large scale acoustic modeling [Electronic resource]. // Fifteenth annual conference of the international speech communication association. --- 2014. --- URL: \url{https://arxiv.org/pdf/1402.1128} (дата обращения: 05.02.2024).

        \bibitem{bib:atten:1}
            Chorowski J. Attention-based models for speech recognition. // Advances in neural information processing systems. --- 2015. --- P. 577-585.

        \bibitem{bib:LM:1}
            Jozefowicz R., Vinyals O., Schuster M., Shazeer N., Wu Y. Exploring the Limits of Language Modeling [Electronic resource]. // arXiv preprint arXiv:1602.02410. --- 2016. --- URL: \url{https://arxiv.org/pdf/1602.02410} (дата обращения: 05.02.2024).

        \bibitem{bib:atten:2}
            Vaswani A., Shazeer N., Parmar N., Uszkoreit J., Jones L. Attention Is All You Need [Electronic resource]. // arXiv preprint arXiv:1706.03762. --- 2023. --- URL: \url{https://arxiv.org/pdf/1706.03762} (дата обращения: 05.02.2024).

    \end{thebibliography}
\endgroup
