\anonsection{перечень сокращений и обозначений}

В настоящей выпускной квалификационной работе применяют следующие обозначения и сокращения.


\begin{itemize}
    \setlength\itemsep{0.8em plus 0.2em minus 0.2em}
    \item[1.]  \textbf{фМРТ} -- функциональная магнитно-резонансная томография.
    \item[2.]  \textbf{ЭМА} -- электромагнитная артикулография.
    \item[3.]  \textbf{ЭКоГ} -- электрокортикография.
    \item[4.]  \textbf{ЭМГ} -- электромиография.
    \item[5.]  \textbf{ЭЭГ} -- электроэнцефалография.


    \item[6.]  \textbf{BRNN} -- bidirectional recurrent neural networks (двунаправленная рекуррентная нейронная сеть).
    \item[7.]  \textbf{CER} -- character error rate (частота символьных ошибок).
    \item[8.]  \textbf{CNN} -- convolutional neural network (cверточная нейронная сеть).
    \item[9.]  \textbf{CTC} -- connectionist temporal classification (коннекционистская временная классификация).
    \item[10.] \textbf{DenseNet} -- densely connected convolutional networks (плотно связанные сверточные сети).
    \item[11.] \textbf{FT} -- fourier transform (преобразование Фурье).
    \item[12.] \textbf{FR} -- frequency ratio (отношение частот).
    \item[13.] \textbf{GRU} -- gated recurrent unit (закрытый рекуррентный модуль).
    \item[14.] \textbf{LSTM} -- long short-term memory (длительная кратковременная память).
    \item[15.] \textbf{MAV} -- mean absolute value (среднее абсолютное значение).
    \item[16.] \textbf{MDF} -- median frequency (медианная частота).
    \item[17.] \textbf{MNF} -- mean frequency (сумма интенсивности спектра).
    \item[18.] \textbf{MPF} -- mean power frequency (средний спектр мощности).
    \item[19.] \textbf{NLP} -- natural language processing (обработка естественного языка).
    \item[20.] \textbf{PF} -- peak frequency (пиковая частота).
    \item[21.] \textbf{ReLU} -- rectified linear unit (выпрямленный линейный блок).
    \item[22.] \textbf{ResNet} -- residual neural network (остаточная нейронная сеть).
    \item[23.] \textbf{RIL} -- relative information lost (относительная потеря информации).
    \item[24.] \textbf{RMS} -- root mean square (среднеквадратичное значение).
    \item[25.] \textbf{RNN} -- recurrent networks (рекуррентные нейронные сети).
    \item[26.] \textbf{SSC} -- slope sign change (изменение знака наклона).
    \item[27.] \textbf{SSI} -- simple square integral (простой квадратный интеграл).
    \item[28.] \textbf{STFT} -- short-time Fourier transform (кратковременное преобразование Фурье).
    \item[29.] \textbf{TPS} -- total power spectrum (общий спектр мощности).
    \item[30.] \textbf{VAR} -- variance (дисперсия).
    \item[31.] \textbf{VCF} -- variance of central frequency (отклонение центральной частоты).
    \item[32.] \textbf{WER} -- word error rate (частота ошибок в словах).
    \item[33.] \textbf{WL} -- wave length (длина волны).
    \item[34.] \textbf{ZC} -- zero crossing (пересечение нуля). 
\end{itemize}
